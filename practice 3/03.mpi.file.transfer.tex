\documentclass[a4paper,11pt]{article}

\usepackage[margin=2.5cm]{geometry}
\usepackage[T1]{fontenc}
\usepackage[utf8]{inputenc}
\usepackage{lmodern}
\usepackage{hyperref}
\usepackage{listings}
\usepackage{xcolor}

\lstset{
  basicstyle=\ttfamily\small,
  breaklines=true,
  frame=single,
  keywordstyle=\color{blue},
  commentstyle=\color{gray},
  showstringspaces=false,
}

\title{Practical Work 3: MPI File Transfer}
\author{
  Doan Dinh Khai - 22BA13167
}
\date{\today}

\begin{document}
\maketitle

\section*{MPI Implementation Choice}

Explain here \emph{which} MPI implementation you decided to use (e.g.\ MPICH, Open~MPI, MS-MPI) and why. Mention:
\begin{itemize}
  \item Operating system and environment you are using.
  \item How you installed MPI and the \texttt{mpi4py} (or other language binding) library.
  \item Advantages or limitations that influenced your choice.
\end{itemize}

\section*{MPI Service Design}

Describe the design of your MPI-based file transfer service. For example:
\begin{itemize}
  \item Which ranks act as \textbf{sender} and \textbf{receiver}.
  \item What metadata (filename, size, etc.) is exchanged.
  \item Whether you send the file as a whole or in chunks.
\end{itemize}

You can include a simple figure of the data flow (e.g.\ using a drawing tool and importing as PDF/PNG) showing how the file moves from the sender process to the receiver process.

\section*{System Organization}

Explain how you organize your code and processes:
\begin{itemize}
  \item Directory structure (e.g.\ where \texttt{mpi\_file\_transfer.py} is located).
  \item How you start the system using \texttt{mpiexec} or \texttt{mpirun}.
  \item Any scripts or helper tools you use to test your implementation.
\end{itemize}

\section*{File Transfer Implementation}

Briefly explain how the actual file transfer is implemented. For example, for the provided Python/\texttt{mpi4py} example:
\begin{itemize}
  \item Rank 0 reads the file, sends metadata (name, size) with \texttt{comm.send}.
  \item Rank 0 then sends the raw bytes in a second message.
  \item Rank 1 receives metadata and bytes, then writes them to disk.
\end{itemize}

Include a short code snippet that represents the core of your MPI transfer logic:

\begin{lstlisting}[language=Python,caption={Core MPI file transfer snippet}]
meta = {"name": filename, "size": file_size}
comm.send(meta, dest=1, tag=0)

with open(filepath, "rb") as f:
    file_bytes = f.read()

comm.send(file_bytes, dest=1, tag=1)
\end{lstlisting}

You can replace this snippet with the exact code you implemented (Python, C, C++, etc.).

\section*{Group Work and Responsibilities}

Describe briefly who did what in your group:
\begin{itemize}
  \item Design and choice of MPI implementation.
  \item Implementation of sender and receiver.
  \item Testing and debugging.
  \item Writing and formatting the \LaTeX{} report.
\end{itemize}

\end{document}


