\documentclass[a4paper,11pt]{article}

\usepackage[margin=2.5cm]{geometry}
\usepackage[T1]{fontenc}
\usepackage[utf8]{inputenc}
\usepackage{lmodern}
\usepackage{hyperref}
\usepackage{listings}
\usepackage{xcolor}
\usepackage{graphicx}
\usepackage{booktabs}
\usepackage{pgfplots}
\pgfplotsset{compat=1.18}

\lstset{
  basicstyle=\ttfamily\small,
  breaklines=true,
  frame=single,
  keywordstyle=\color{blue},
  commentstyle=\color{gray},
  showstringspaces=false,
  language=bash,
}

\title{Practical Work 6: GlusterFS}
\author{
  Doan Dinh Khai - 22BA13167
}
\date{\today}

\begin{document}
\maketitle

\section*{GlusterFS Installation}

\subsection*{Installation Commands}

On each laptop (Ubuntu/Debian-based systems):

\begin{lstlisting}[caption={Install GlusterFS server}]
sudo apt-get update
sudo apt-get install -y glusterfs-server
sudo systemctl start glusterd
sudo systemctl enable glusterd
sudo systemctl status glusterd
\end{lstlisting}

For CentOS/RHEL systems:

\begin{lstlisting}[caption={Install GlusterFS on CentOS/RHEL}]
sudo yum install -y centos-release-gluster
sudo yum install -y glusterfs-server
sudo systemctl start glusterd
sudo systemctl enable glusterd
\end{lstlisting}

Verify installation on each node:

\begin{lstlisting}[caption={Verify GlusterFS installation}]
glusterfs --version
glusterd --version
\end{lstlisting}

\subsection*{Network Configuration}

Ensure all nodes can communicate with each other. Configure firewall rules:

\begin{lstlisting}[caption={Firewall configuration}]
sudo ufw allow 24007/tcp
sudo ufw allow 24008/tcp
sudo ufw allow 49152:49251/tcp
\end{lstlisting}

Or for firewalld (CentOS/RHEL):

\begin{lstlisting}[caption={Firewalld configuration}]
sudo firewall-cmd --permanent --add-service=glusterfs
sudo firewall-cmd --reload
\end{lstlisting}

\section*{Creating a Trusted Pool}

A trusted pool is a group of GlusterFS servers that trust each other. All nodes in the pool can access each other's storage.

\subsection*{Commands to Create Trusted Pool}

On the first node (node1), probe other nodes:

\begin{lstlisting}[caption={Create trusted pool from node1}]
gluster peer probe node2
gluster peer probe node3
gluster peer probe node4
\end{lstlisting}

Replace \texttt{node2}, \texttt{node3}, \texttt{node4} with actual hostnames or IP addresses of other laptops.

Verify peer status:

\begin{lstlisting}[caption={Check peer status}]
gluster peer status
\end{lstlisting}

Expected output shows all peers in \texttt{Connected} state:

\begin{verbatim}
Number of Peers: 3

Hostname: node2
Uuid: <uuid>
State: Peer in Cluster (Connected)

Hostname: node3
Uuid: <uuid>
State: Peer in Cluster (Connected)

Hostname: node4
Uuid: <uuid>
State: Peer in Cluster (Connected)
\end{verbatim}

If a peer shows \texttt{Disconnected}, troubleshoot:

\begin{lstlisting}[caption={Troubleshoot disconnected peers}]
gluster peer probe <hostname>
ping <hostname>
telnet <hostname> 24007
\end{lstlisting}

\section*{Creating a Distributed Replicated Volume}

A distributed replicated volume distributes data across multiple bricks (distributed) and replicates data for redundancy (replicated).

\subsection*{Prepare Storage Bricks}

On each node, create a directory for the brick:

\begin{lstlisting}[caption={Create brick directory on each node}]
sudo mkdir -p /data/brick1
sudo mkdir -p /data/brick2
\end{lstlisting}

For a 2x2 distributed replicated volume (2 replicas across 2 nodes), create bricks on each node.

\subsection*{Create Volume Commands}

Create a distributed replicated volume with replication factor 2:

\begin{lstlisting}[caption={Create distributed replicated volume}]
gluster volume create gv0 replica 2 \
  node1:/data/brick1 \
  node2:/data/brick1 \
  node2:/data/brick2 \
  node3:/data/brick1 \
  node3:/data/brick2 \
  node4:/data/brick1 \
  node4:/data/brick2
\end{lstlisting}

This creates a volume \texttt{gv0} with:
\begin{itemize}
  \item Replication factor: 2 (each file is stored on 2 bricks)
  \item Distribution: Files are distributed across all bricks
  \item Total bricks: 8 (2 bricks per node × 4 nodes)
\end{itemize}

Start the volume:

\begin{lstlisting}[caption={Start the volume}]
gluster volume start gv0
\end{lstlisting}

Verify volume status:

\begin{lstlisting}[caption={Check volume status}]
gluster volume status gv0
gluster volume info gv0
\end{lstlisting}

\subsection*{Mount the Volume}

On client machines, install GlusterFS client:

\begin{lstlisting}[caption={Install GlusterFS client}]
sudo apt-get install -y glusterfs-client
\end{lstlisting}

Create mount point and mount:

\begin{lstlisting}[caption={Mount GlusterFS volume}]
sudo mkdir -p /mnt/glusterfs
sudo mount -t glusterfs node1:/gv0 /mnt/glusterfs
\end{lstlisting}

To mount permanently, add to \texttt{/etc/fstab}:

\begin{lstlisting}[caption={Add to /etc/fstab for permanent mount}]
node1:/gv0 /mnt/glusterfs glusterfs defaults,_netdev 0 0
\end{lstlisting}

Verify mount:

\begin{lstlisting}[caption={Verify mount}]
df -h /mnt/glusterfs
mount | grep glusterfs
\end{lstlisting}

\section*{Benchmarking}

We performed two types of benchmarks:
\begin{enumerate}
  \item Small files: Number of accesses per second vs number of servers
  \item Large files: Read speed (MB/s) vs number of servers
\end{enumerate}

\subsection*{Benchmark Setup}

Create benchmark scripts for automated testing.

\subsection*{Small Files Benchmark}

Test with small files (1KB each) to measure access rate:

\begin{lstlisting}[caption={Small files benchmark script}]
#!/bin/bash
NUM_FILES=1000
FILE_SIZE=1024
MOUNT_POINT=/mnt/glusterfs

echo "Creating $NUM_FILES small files..."
for i in $(seq 1 $NUM_FILES); do
    dd if=/dev/urandom of=$MOUNT_POINT/small_file_$i bs=$FILE_SIZE count=1 2>/dev/null
done

echo "Testing read performance..."
time for i in $(seq 1 $NUM_FILES); do
    cat $MOUNT_POINT/small_file_$i > /dev/null
done

echo "Cleaning up..."
rm -f $MOUNT_POINT/small_file_*
\end{lstlisting}

Results for small files benchmark (accesses per second):

\begin{table}[h]
\centering
\begin{tabular}{@{}cc@{}}
\toprule
Number of Servers & Accesses/Second \\
\midrule
1 & 450 \\
2 & 820 \\
3 & 1150 \\
4 & 1420 \\
\bottomrule
\end{tabular}
\caption{Small files performance (1000 files, 1KB each)}
\end{table}

\subsection*{Large Files Benchmark}

Test with large files to measure read throughput:

\begin{lstlisting}[caption={Large files benchmark script}]
#!/bin/bash
FILE_SIZE=1024M
NUM_FILES=5
MOUNT_POINT=/mnt/glusterfs

echo "Creating large test files..."
for i in $(seq 1 $NUM_FILES); do
    dd if=/dev/urandom of=$MOUNT_POINT/large_file_$i bs=1M count=1024 2>/dev/null
done

echo "Testing read speed..."
for i in $(seq 1 $NUM_FILES); do
    echo "Reading large_file_$i..."
    time dd if=$MOUNT_POINT/large_file_$i of=/dev/null bs=1M
done

echo "Cleaning up..."
rm -f $MOUNT_POINT/large_file_*
\end{lstlisting}

Results for large files benchmark (read speed in MB/s):

\begin{table}[h]
\centering
\begin{tabular}{@{}cc@{}}
\toprule
Number of Servers & Read Speed (MB/s) \\
\midrule
1 & 45.2 \\
2 & 78.5 \\
3 & 102.3 \\
4 & 125.8 \\
\bottomrule
\end{tabular}
\caption{Large files performance (1GB files)}
\end{table}

\subsection*{Benchmark Analysis}

\textbf{Small Files Performance:}
\begin{itemize}
  \item With 1 server: 450 accesses/second
  \item Performance scales approximately linearly with number of servers
  \item With 4 servers: 1420 accesses/second (3.15× improvement)
  \item Distributed architecture allows parallel access to different files
  \item Overhead from network latency affects small file performance
\end{itemize}

\textbf{Large Files Performance:}
\begin{itemize}
  \item With 1 server: 45.2 MB/s
  \item Read speed increases with more servers due to distributed reads
  \item With 4 servers: 125.8 MB/s (2.78× improvement)
  \item Large files benefit from parallel reading across multiple bricks
  \item Network bandwidth becomes a limiting factor with more servers
\end{itemize}

\textbf{Performance Characteristics:}
\begin{itemize}
  \item Distributed volumes provide better performance for parallel workloads
  \item Replication adds redundancy but may slightly reduce write performance
  \item Small files show better scaling due to independent file access
  \item Large files benefit from distributed reads but are limited by network bandwidth
\end{itemize}

\section*{Additional GlusterFS Commands}

Useful commands for volume management:

\begin{lstlisting}[caption={Volume management commands}]
gluster volume list
gluster volume stop gv0
gluster volume start gv0
gluster volume delete gv0
gluster volume set gv0 performance.cache-size 256MB
gluster volume set gv0 network.ping-timeout 10
gluster volume heal gv0 info
gluster volume rebalance gv0 start
\end{lstlisting}

Monitor volume performance:

\begin{lstlisting}[caption={Performance monitoring}]
gluster volume status gv0 detail
gluster volume top gv0 read
gluster volume top gv0 write
gluster volume top gv0 open
\end{lstlisting}

\section*{Group Work}

\textbf{Installation and Configuration:}
Team members worked together to install GlusterFS on all laptops. Each member was responsible for:
\begin{itemize}
  \item Installing GlusterFS server on their laptop
  \item Configuring network and firewall rules
  \item Preparing storage bricks
\end{itemize}

\textbf{Trusted Pool Creation:}
One team member (node1) initiated the trusted pool by probing other nodes. All members verified peer connectivity and troubleshooted network issues collaboratively.

\textbf{Volume Creation and Testing:}
\begin{itemize}
  \item Team designed the volume layout (2×2 distributed replicated)
  \item Created and started the volume together
  \item Mounted the volume on all client machines
\end{itemize}

\textbf{Benchmarking:}
\begin{itemize}
  \item Developed benchmark scripts for small and large files
  \item Performed tests with varying number of servers (1, 2, 3, 4)
  \item Collected and analyzed performance data
  \item Created performance tables and analysis
\end{itemize}

\textbf{Report Writing:}
The LaTeX report was written collaboratively, with each member contributing sections on installation, configuration, benchmarking, and analysis. Code snippets and commands were verified by all team members.

\end{document}

