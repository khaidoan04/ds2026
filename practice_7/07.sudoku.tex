\documentclass[a4paper,11pt]{article}

\usepackage[margin=2.5cm]{geometry}
\usepackage[T1]{fontenc}
\usepackage[utf8]{inputenc}
\usepackage{lmodern}
\usepackage{hyperref}
\usepackage{listings}
\usepackage{xcolor}
\usepackage{graphicx}

\lstset{
  basicstyle=\ttfamily\small,
  breaklines=true,
  frame=single,
  keywordstyle=\color{blue},
  commentstyle=\color{gray},
  showstringspaces=false,
}

\title{Practical Work 7: Sudoku on Google App Engine (GAE)}
\author{
  Doan Dinh Khai - 22BA13167
}
\date{\today}

\begin{document}
\maketitle

\section*{Project Overview}

This practical work implements a simple Sudoku web application deployed on
Google App Engine (GAE) using Python and Flask. The application allows users
to input a Sudoku puzzle, solve it on the server side, and display the
completed grid in the browser.

\section*{Application Structure}

The project is organised as follows:

\begin{lstlisting}[language=bash,caption={Project structure}]
practice_7/
  app.yaml
  main.py
  requirements.txt
  07.sudoku.tex
\end{lstlisting}

\begin{itemize}
  \item \texttt{main.py}: Flask application and Sudoku solver logic.
  \item \texttt{app.yaml}: GAE configuration file (runtime and entrypoint).
  \item \texttt{requirements.txt}: Python dependencies (Flask).
  \item \texttt{07.sudoku.tex}: LaTeX report for this practical work.
\end{itemize}

\section*{Google App Engine Configuration}

The \texttt{app.yaml} file defines the runtime and entrypoint:

\begin{lstlisting}[language=yaml,caption={app.yaml}]
runtime: python312

entrypoint: gunicorn -b :$PORT main:app

handlers:
  - url: /.*
    script: auto
\end{lstlisting}

This configuration:
\begin{itemize}
  \item Uses the Python 3.12 runtime.
  \item Starts the Flask application via Gunicorn.
  \item Routes all URLs to the application.
\end{itemize}

Python dependencies are declared in \texttt{requirements.txt}:

\begin{lstlisting}[language=bash,caption={requirements.txt}]
Flask==3.0.0
\end{lstlisting}

\section*{Sudoku Application Design}

\subsection*{Flask Application}

The application is built with a single Flask instance:

\begin{lstlisting}[language=Python,caption={Flask setup}]
from flask import Flask, render_template_string, request

app = Flask(__name__)
\end{lstlisting}

The root route \texttt{/} handles both \texttt{GET} and \texttt{POST} requests.
On \texttt{GET}, it displays an empty Sudoku grid. On \texttt{POST}, it either
solves the puzzle or clears the board, depending on the button clicked.

\subsection*{Board Representation}

The Sudoku board is represented as a 9×9 matrix of integers:

\begin{lstlisting}[language=Python,caption={Empty board}]
EMPTY_BOARD = [
    [0, 0, 0, 0, 0, 0, 0, 0, 0],
    [0, 0, 0, 0, 0, 0, 0, 0, 0],
    [0, 0, 0, 0, 0, 0, 0, 0, 0],
    [0, 0, 0, 0, 0, 0, 0, 0, 0],
    [0, 0, 0, 0, 0, 0, 0, 0, 0],
    [0, 0, 0, 0, 0, 0, 0, 0, 0],
    [0, 0, 0, 0, 0, 0, 0, 0, 0],
    [0, 0, 0, 0, 0, 0, 0, 0, 0],
    [0, 0, 0, 0, 0, 0, 0, 0, 0],
]
\end{lstlisting}

User input from the HTML form is converted to this matrix using
\texttt{parse\_board()}, which reads each cell, validates digits 1–9, and
stores zeros for empty or invalid entries.

\section*{Sudoku Solver Logic}

The solver uses a classic backtracking algorithm with three main functions:

\begin{itemize}
  \item \texttt{is\_valid(board, row, col, num)}: Checks if placing a number in
        a specific cell is valid according to Sudoku rules (row, column,
        and 3×3 subgrid).
  \item \texttt{solve\_sudoku(board)}: Recursively searches for a solution by
        trying numbers 1–9 in each empty cell.
  \item \texttt{board\_to\_strings(board)}: Converts the numeric board into
        strings for display in the HTML form.
\end{itemize}

\begin{lstlisting}[language=Python,caption={Validation and solver}]
def is_valid(board, row, col, num):
    for c in range(9):
        if board[row][c] == num:
            return False
    for r in range(9):
        if board[r][col] == num:
            return False
    start_row = (row // 3) * 3
    start_col = (col // 3) * 3
    for r in range(start_row, start_row + 3):
        for c in range(start_col, start_col + 3):
            if board[r][c] == num:
                return False
    return True

def solve_sudoku(board):
    for row in range(9):
        for col in range(9):
            if board[row][col] == 0:
                for num in range(1, 10):
                    if is_valid(board, row, col, num):
                        board[row][col] = num
                        if solve_sudoku(board):
                            return True
                        board[row][col] = 0
                return False
    return True
\end{lstlisting}

\section*{User Interface}

The HTML interface is defined as a template string rendered with
\texttt{render\_template\_string()}. The grid is built as a 9×9 table with
input fields:

\begin{itemize}
  \item Each cell is named \texttt{cell-r-c} where \texttt{r} and \texttt{c}
        are row and column indices.
  \item Two buttons are provided: \textbf{Solve} and \textbf{Clear}.
  \item Simple CSS styling highlights 3×3 subgrids and centers the board.
  \item A message area displays errors (e.g. unsolvable puzzles).
\end{itemize}

On \textbf{Solve}, the server parses the board, runs the solver, and returns
the completed grid. On \textbf{Clear}, the board is reset to an empty state.

\section*{Deployment on Google App Engine}

Steps to deploy the application:

\begin{enumerate}
  \item Create a new project on \texttt{https://console.cloud.google.com}.
  \item Install and configure the Google Cloud SDK.
  \item Navigate to the \texttt{practice\_7} directory:
    \begin{lstlisting}[language=bash]
gcloud init
gcloud auth login
cd practice_7
gcloud app deploy
    \end{lstlisting}
  \item Open the application in a browser:
    \begin{lstlisting}[language=bash]
gcloud app browse
    \end{lstlisting}
\end{enumerate}

GAE automatically installs dependencies from \texttt{requirements.txt} and
starts the application according to \texttt{app.yaml}.

\section*{Group Work}

\textbf{Project Structure and Design:}
The group discussed and agreed on using Flask on GAE for a lightweight
Sudoku web application. One member designed the application structure and
configured the GAE deployment files (\texttt{app.yaml}, \texttt{requirements.txt}).

\textbf{Implementation:}
The Sudoku solver and Flask routes were implemented collaboratively. The
backtracking algorithm was tested with multiple puzzles to ensure correctness.
The HTML/CSS layout was refined to provide a clean and simple user interface.

\textbf{Testing and Deployment:}
Group members tested the application locally using the Flask development
server and then deployed it to GAE using the Google Cloud SDK. Each member
verified that the application worked from their browser and that puzzles were
solved correctly.

\textbf{Report Writing:}
This report was written in \LaTeX{}, documenting the application structure,
implementation details, important code snippets, and group responsibilities.
All members reviewed the report before submission.

\end{document}


